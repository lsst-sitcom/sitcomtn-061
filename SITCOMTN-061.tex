\documentclass[SE,authoryear,toc,lsstdraft]{lsstdoc}
\input{meta}

% Package imports go here.

% Local commands go here.

%If you want glossaries
%\input{aglossary.tex}
%\makeglossaries

\title{System First Light Definition}

% Optional subtitle
% \setDocSubtitle{A subtitle}

\author{%
Keith Bechtol, Anastasia Alexov, Chuck Claver, Leanne Guy, Robert Lupton, Kevin Reil, Steve Ritz, Austin Roberts
}

\setDocRef{SITCOMTN-061}
\setDocUpstreamLocation{\url{https://github.com/lsst-sitcom/sitcomtn-061}}

\date{\vcsDate}

% Optional: name of the document's curator
% \setDocCurator{The Curator of this Document}

\setDocAbstract{%
System First Light (SFL) is an intermediate technical milestone between recording the first on-sky images with LSSTCam and the start of the 10-year LSST survey.
The SFL milestone is met when we can routinely acquire science-grade imaging across the full focal plane, and have a well understood technical path towards meeting the Construction Completeness criteria.
This document provides a technical definition of the SFL milestone, and describes an example test dataset that could be used to verify that these criteria have been satisfied.
%The System First Light (SFL) milestone
%to clarify and the expected state of the system and relationship to other Construction milestones.
}

% Change history defined here.
% Order: oldest first.
% Fields: VERSION, DATE, DESCRIPTION, OWNER NAME.
% See LPM-51 for version number policy.
\setDocChangeRecord{%
  \addtohist{1}{YYYY-MM-DD}{Unreleased.}{Keith Bechtol}
}


\begin{document}

% Create the title page.
\maketitle
% Frequently for a technote we do not want a title page  uncomment this to remove the title page and changelog.
% use \mkshorttitle to remove the extra pages

% ADD CONTENT HERE
% You can also use the \input command to include several content files.

\section{Overview}
\label{overview}

\textbf{System First Light} (SFL) is an intermediate technical milestone between recording the first on-sky images with LSSTCam and the start of the 10-year LSST survey.
\emph{The SFL milestone is met when we can routinely acquire science-grade imaging across the full focal plane}, and have a well understood technical path towards meeting the Construction Completeness criteria \citedsp{SITCOMTN-005}.
The SFL milestone represents

\begin{itemize}
  \item formal retirement of major technical risks to overall system performance (including delivered image quality and system throughput),
  \item reduced uncertainty on the remaining Construction schedule, and
  \item an increased emphasis on bulk data collection to support science verification and validation activities leading up to the Operations Readiness Review.
\end{itemize}

Section \ref{criteria} provides a set of quantitative criteria to define the SFL milestone, and Section \ref{dataset} describes an example test dataset.

The System First Light milestone will be used by Rubin Observatory Operations to more accurately gauge the start date of the 10-year LSST survey and to prepare for Early Science, including the Data Previews and start of Alert Production \citedsp{RTN-011} \href{https://rtn-011.lsst.io/}.
As of early 2023, it is expected that Data Preview 1 will be based on a subset of science-grade images taken with LSSTCam during a period of a few days around the System First Light milestone.
System First Light is also a widely publicized milestone for external stakeholders (e.g., \href{https://ls.st/dates}{ls.st/dates}), with anticipated celebratory events, media coverage, and release of images to preview the scientific potential of Rubin Observatory (e.g., color coadd images and difference images to illustrate the detection of transient, variable, and/or moving objects).

In contrast to SFL, \textbf{First Photon} is the milestone corresponding to first on-sky photons passing through the fully integrated optical system and being recorded with LSSTCam.
First Photon is primarily a functional demonstration, whereas SFL includes additional criteria on science performance and fitness for bulk data collection.
%and is understood as primarily a functional demonstration.
%By comparison, the System First Light milestone is met when we can routinely acquire science-grade imaging across the full focal plane and have a well understood technical path towards meeting the Construction Completeness criteria \citedsp{SITCOMTN-005} with minimal technical or schedule risks.

\section{System First Light Criteria}
\label{criteria}

By the System First Light milestone, we are confident that

\begin{enumerate}
  \item members of the commissioning team (not necessarily Operations team) are able to observe any set of target fields with airmass $\leq 2$ and acquire science images for at least an hour without interruption at a cadence similar to the baseline LSST survey (i.e., $\sim90$ visits per hour with each visit corresponding to a standard 30-second total exposure),
  %point the telescope to any right ascension and declination $(\alpha, \delta)$ coordinate with airmass $\leq 2$ and acquire science images for at least 1 hour without interruption,
  %\item members of the commissioning team (not necessarily Operations team) are able to point the telescope to any right ascension and declination $(\alpha, \delta)$ coordinate with airmass $\leq 2$ and acquire science images for at least 1 hour without interruption,
  \item those data can be characterized using metrics defined for Priority 1a requirements from the LSR \citedsp{LSE-29} and OSS \citedsp{LSE-30} related to single-visit image quality in the $r$ band + at least two of the $giz$ bands,
  \item those data meet the following specifications for intrinsic system performance to be acceptable for LSST science:
  \begin{enumerate}
    \item median system contribution to delivered image quality (adjusted to zenith) less than 0.7 arcseconds across the full focal plane for $r$- and $i$-band images
    %\footnote{For comparison, the SRD minimum specification (Table 9) corresponds a 0.4 arcsecond system contribution to delivered image quality.
    %To achieve SFL, the active optics system (AOS) will control the figures of M1M3 and M2 optical surfaces and the camera focus and alignment at a range of pointing elevations.
    %System optimization including the AOS, dome environment, TMA tracking, etc., to surpass the SRD design specification for delivered image quality is anticipated to continue between the SFL milestone and start of LSST 10-year survey.}
    \item median system throughput integral (atmosphere removed or using a standard atmosphere) known to within 10\% uncertianty to meet minimum specifications defined in the SRD (expressed as image depth for point sources as in Table 6) converted to instrumental zeropoints \citedsp{SMTN-002},
    \item performance of the LSSTCam as installed on the TMA is consistent with expectations based on acceptance testing at SLAC and re-verification on Level 3, and consistent with supporting SRD requirements (e.g., system read noise, usable pixel fraction, dynamic range),
  \end{enumerate}
  \item system telemetry, raw and basic ISR-corrected images, as well as diagnostics for single-visit science performance sampled at multiple locations on the focal plane are visible to observers on the summit within 5 minutes of acquiring the visits,
  \item following automated data transfer and ingest at USDF, the data management system is functionally capable of both building a 3-band color coadd image and producing difference images from these data that have sufficient information content to support creation of ``press-ready'' images,
  \item there exists a well understood technical path towards satisfying the Construction Completeness criteria \citedsp{SITCOMTN-005} with minimal technical or schedule risks, including meeting all performance requirements in the SRD at their minimum specifications or better during the LSST 10-year survey.
  %Priority 1 science requirements by the start of LSST 10-year survey at the minimum thresholds specified in the SRD in all requirements.
\end{enumerate}

In order to achieve the milestone, the commissioning team should have reasonable expectation that the criteria above can be met on any given night, for example, based on three or more consecutive nights of stable performance when observing fields at a range of airmass from 1.0 to 2.0.
This level of routine functional and science performance, with acceptable observing efficiency, is needed to support bulk data collection to complete commissioning science verification and validation, e.g., the Science Validation survey.

As part of establishing a well understood technical path towards satisfying the Construction Completeness criteria, at least a subset of $r$- and $i$-band visits prior to SFL (spanning a range of airmass from 1.0 to 2.0) should have system contribution to delivered image quality (adjusted to zenith) less than 0.4 arcseconds across the full focal plane to demonstrate that the as-built hardware and software is \emph{capable} of meeting minimum SRD specifications with appropriate control of the AOS, telescope tracking, dome environment, etc., even if this level of performance is not yet \emph{routinely} achieved at the time of SFL.

The commissioning plan includes a period of system optimization following SLF that is intended to enhance the stability of the system across a range of observing conditions.
For the baseline AOS commissioning plan, we expect that validation of the open-loop (look-up table) and closed-loop systems considering the full set of degrees of freedom can utilize the same set of imaging data collected for the science validation survey (except for a brief period of full array mode out-of-focus data to validate the sensitivity matrix).

\section{Example Test Dataset to Support System First Light Milestone}
\label{dataset}

\subsection{Minimal Dataset}

The criteria listed in Section \ref{criteria} could be verified with the following example ``mimimal'' test dataset that could be acquired in roughly an hour of on-sky observations (roughly $\sim90$ visits, assuming each visit is $\sim40$ seconds for integration, readout, and slew) taken on each of three nights.
Take repeated observations of a field (e.g., 30 visits in each of the $gri$ bands) with translational dithers ranging from detector-scale up to focal-plane scale so that individual stars are measured on multiple detectors across the focal plane.
The number of visits is chosen to approximate the number of overlapping visits that would be acquired during the first year of the LSST Wide-Fast-Deep survey over a region of $\sim10$ square degrees.
For a typical high Galactic latitude density of $\sim1$ Gaia reference star per square arcminute, this region would include measurements of roughly $\sim3 \times 10^4$ high SNR stars, and $\sim1$ million individual measurements of the PSF sampled across the focal plane.
For the minimal dataset, the observing epochs spread across multiple nights do not necessarily need to cover the same field, but should span a range of airmass, e.g., 1.0, 1.4, and 2.0, to demonstrate stable performance.

\subsection{Baseline Dataset}

In practice, we plan that the minimal set of observations above would be acquired as part of a larger ``dense dithered star field'' science program with multiple fields, observed in at least 5 bands, in multiple epochs spread across several nights.
For example, 4 fields $\times$ 30 visits $\times$ 5 bands $\times$ 3 epochs with 600 visits per night could be acquired in 3 full nights of observing, or 6 half-nights of observing.
The epochs and fields should be selected to sample observations at a range of airmass, e.g., 1.0, 1.4, and 2.0, with the repeated epochs for a given field scheduled on distinct nights to sample a larger time baseline.
By slewing between multiple fields spanning a range of telescope elevations within a single night, we could verify the AOS performance to maintain the figure of optical surfaces for gravitational forces at a range of telescope orientations.
30 visits of the same field in the same band within a single epoch is desireable to achieve stable environmental conditions and isolate effects related to the instrumental calibration (e.g., illumination correction, camera distortion model).
Scheduling multiple epochs of the same fields allows for tests of difference imaging and characterization of performance across a range of environmental conditions.

The larger ``baseline'' dataset would enable multiple science verification studies, including
\begin{itemize}
  \item determination of illumination correction, photometric repeatability, and chromatic part of instrumental throughput (same star through multiple detectors),
  \item determination of camera distortion model, astrometric repeatability, relationship between DCR and chromatic PSF, and
  \item initial testing of coaddition and difference imaging.
\end{itemize}

A subset of these images could form the basis of Data Preview 1 \citedsp{RTN-011}.

\appendix
% Include all the relevant bib files.
% https://lsst-texmf.lsst.io/lsstdoc.html#bibliographies
\section{References} \label{sec:bib}
\renewcommand{\refname}{} % Suppress default Bibliography section
\bibliography{local,lsst,lsst-dm,refs_ads,refs,books}

% Make sure lsst-texmf/bin/generateAcronyms.py is in your path
\section{Acronyms} \label{sec:acronyms}
\addtocounter{table}{-1}
\begin{longtable}{p{0.145\textwidth}p{0.8\textwidth}}\hline
\textbf{Acronym} & \textbf{Description}  \\\hline

AOS & Active Optics System \\\hline
DCR & Differential Chromatic Refraction \\\hline
ISR & Instrument Signal Removal \\\hline
LSE & LSST Systems Engineering (Document Handle) \\\hline
LSR & LSST System Requirements; LSE-29 \\\hline
LSST & Legacy Survey of Space and Time (formerly Large Synoptic Survey Telescope) \\\hline
M1M3 & Primary Mirror Tertiary Mirror \\\hline
M2 & Secondary Mirror \\\hline
OSS & Observatory System Specifications; LSE-30 \\\hline
PSF & Point Spread Function \\\hline
RTN & Rubin Technical Note \\\hline
SE & System Engineering \\\hline
SLAC & SLAC National Accelerator Laboratory \\\hline
SNR & Signal to Noise Ratio \\\hline
SRD & LSST Science Requirements; LPM-17 \\\hline
TMA & Telescope Mount Assembly \\\hline
USDF & United States Data Facility \\\hline
\end{longtable}

% If you want glossary uncomment below -- comment out the two lines above
%\printglossaries





\end{document}
